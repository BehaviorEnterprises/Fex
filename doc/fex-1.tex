\documentclass{latex2man}

\begin{Name}{1}{fex}{Jesse McClure}{Frequency Excusion}{FEX --- The frequency excursion calculator}

\Prog{Fex} - The frequency excursion calculator

\end{Name}

\section{Description}

\Prog{Fex} is a bioacoustic analysis tool for calculating the frequency
excursion of an audio recording of bird songs or other sound waves.

\section{Fex-gtk}

\Prog{Fex-gtk} is a wrapper script for fex that facilitates batch
processing of large numbers of wave files.  \Prog{Fex-gtk} can be
provided a list of file names, or can be run with no parameters to
provide a dialog window to select input wave files.  \Prog{Fex} will be
run on each input file, and the results can either be stored to a data
file or displayed in a dialog window upon completion.

\Prog{Fex-gtk} is also the executable target of \File{fex.desktop} to
allow for drag-and-drop operation of \Prog{fex}.  The desktop file can
accept any number of wave files as a drop target.

\section{Configuration}
\Prog{Fex} is configured via a runtime configuration file read from one
of
\File{\$XDG\_CONFIG\_HOME/fex/config},
\File{\$HOME/.config/fex/config},
\File{\$HOME/.fexrc}, or
\File{/usr/share/fex/config}.

Read the example configuration file at \File{/usr/share/fex/config} for
a complete description of the available options.  Those listed below are
only the most likely to be changed.  Settings not listed below should be
changed only with caution.

\subsection{Settings}

The options below are configured with the following format:

\Opt{set} \Arg{option} = \Arg{value}

\begin{description}
\item[\Opt{threshold}]
	Amplitude at which points can be considered part of the signal
	in dB below the maximum amplitude.
\item[\Opt{floor}]
	Sound floor for the spectrogram display
	in dB below the maximum amplitude.
\item[\Opt{bandpass}]
	Hi pass and low pass filter values in kHz.
\end{description}


\subsection{Colors}

Colors are configured with the following format where each setting is a
floating point value between 0.00 and 1.00:

\Opt{color} \Arg{element} = \Arg{Red} \Arg{Green} \Arg{Blue} \Arg{Alpha}
\Arg{Width}

\begin{description}
\item[\Opt{spectrogram}]
	Background spectrogram
\item[\Opt{threshold}]
	Points above the current threshold
\item[\Opt{points}]
	Points included in the current frequency excursion calculation
\item[\Opt{lines}]
	Lines connecting the peak points
\end{description}

\section{Author}
Copyright \copyright 2013-2014 Jesse McClure \\
License GPLv3: GNU GPL version 3 \URL{http://gnu.org/licenses/gpl.html} \\
This is free software: you are free to change and redistribute it. \\
There is NO WARRANTY, to the extent permitted by law.

Submit bug reports via github: \\
\URL{http://github/com/TrilbyWhite/fex.git}

I would like your feedback.  If you benefit from \Prog{Fex} see the
bottom of the site below for detauls on submitting comments: \\
\URL{http://mccluresk9.com/software.html}

\LatexManEnd
